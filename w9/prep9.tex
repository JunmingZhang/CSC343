\documentclass{article}
\usepackage[utf8]{inputenc}

\usepackage[dvipsnames]{xcolor}
\usepackage{tcolorbox}

\usepackage{array}
\usepackage{wrapfig}
\usepackage{multirow}
\usepackage{tabu}

\usepackage{amsmath}

\usepackage{geometry}   %设置页边距的宏包
\geometry{left=3cm,right=3cm,top=2.5cm,bottom=2.5cm}  %设置 上、左、下、右 页边距

\usepackage{colortbl}

\newtcolorbox{mybox}{colback = white, colframe = blue!75!white}

\title{CSC343 Prep 9}
\author{Junming Zhang}
\date{July 2019}

\begin{document}

\maketitle
\noindent
{\color{red} \rule{\linewidth}{0.25mm} }

\begin{enumerate}
    \item Suppose we have a relation on attributes $A, B, C, D, E,$ and $F$ , and these functional dependencies hold: $S=\{B \rightarrow DE, BF \rightarrow C, CF \rightarrow B, DF \rightarrow AE\}$.
    \begin{enumerate}
        \item Compute $B^+$.
        \[\textcolor{blue}{B^+ = BDE}\]
        \item Compute $CF^+$.
        \[\textcolor{blue}{CF^+ = ABCDEF}\]
        \item Compute $DF^+$.
        \[\textcolor{blue}{DF^+ = ADEF}\]
        \item Compute $BC^+$.
        \[\textcolor{blue}{BC^+ = BCDE}\]
        \item Compute $ABC^+$.
        \[\textcolor{blue}{ABC^+ = ABCDE}\]
    \end{enumerate}
    
    \item Again, suppose we have a relation on attributes $A,B,C,D,E$, and $F$,and these functional dependencies hold: $S = \{B \rightarrow DE, BF \rightarrow C, CF \rightarrow B, DF \rightarrow AE\}$.
    \\Write “yes” or “no” for each, and show your rough work.
    \begin{enumerate}
        \item Does it follow from $S$ that $B \rightarrow A$?
        \\
        \textcolor{blue}{No, because $B^+$ does not include $A$ from $Q1$ part $(a)$.}
        \item Does it follow from $S$ that $CF \rightarrow E$?
        \\
        \textcolor{blue}{Yes, because $CF^+$ includes $E$ from $Q1$ part $(b)$.}
        \item Does it follow from $S$ that $DF \rightarrow B$?
        \\
        \textcolor{blue}{No, because $DF+$ does not include $B$ from $Q1$ part $(c)$.}
        \item Does it follow from $S$ that $BD \rightarrow C$?
        \\
        \textcolor{blue}{The closure of $BD$ can be derived from the functional dependencies given, which is}
        \[\textcolor{blue}{BD^+ = BDE.}\]
        \textcolor{blue}{No, C can not be derived from $BD^+$ since $BD^+$ does not include $C$.}
        \item Does it follow from $S$ that $BFC \rightarrow A$?
        \\
        \textcolor{blue}{The closure of $BFC$ can be derived from the functional dependencies given, which is}
        \[\textcolor{blue}{BFC^+ = ABCDEF.}\]
        \textcolor{blue}{Yes, A can be derived from $BFC^+$ since $BFC^+$ includes $A$.}
    \end{enumerate}
    \item Suppose we have a relation with attributes $ABCDE$ and these functional dependencies: $S = \{ A \rightarrow D, B \rightarrow A, C \rightarrow A, D \rightarrow CE. \}$ Project the functional dependencies onto the attribute set $ABD$.
    \\
    \begin{mybox}
    \textcolor{blue}{By the algorithm introduced in the lecture, compute $A^+, B^+, D^+$ at first, which is,}
    \begin{gather*}
        \textcolor{blue}{A^+ = ACDE}\\
        \textcolor{blue}{B^+ = ABCDE}\\
        \textcolor{blue}{D^+ = ACDE}
    \end{gather*}
    \textcolor{blue}{Therefore, the aimed dependencies are computed, }
    \begin{gather*}
        \textcolor{blue}{A \rightarrow D}\\
        \textcolor{blue}{B \rightarrow AD}\\
        \textcolor{blue}{D \rightarrow A}
    \end{gather*}
    \textcolor{blue}{Since $B$ determines all behaviors of $A, B$ and $D$ by $B^+$, there is no need to consider any superset with $B$. Also,}
    \[\textcolor{blue}{AD^+ = ACDE,}\]
    \textcolor{blue}{which does not contribute any new functional dependencies on attributes $A, B$, and $D$.}\\
    \textcolor{blue}{Finally, the projection is solved as,}
    \[\textcolor{blue}{\{A \rightarrow D, B \rightarrow AD, D \rightarrow A\}.}\]
    \end{mybox}

    \item Consider relation $R(A, B, C, D, E, F )$ with functional dependencies: \[S = \{CD \rightarrow A, B \rightarrow EF, A \rightarrow BC, F \rightarrow D\}\]
    Create an instance of $R$ that satisfies its FDs and has redundant data. Identify redundancy by circling a single value in the table that could be erased and yet we would know what its value \textit{must} be. Thought exercise: what does it have to do with the FDs?
    
        \begin{table}[h!]
        \centering
        \color{blue}\begin{tabular}{|| c || c || c || c || c || c||} 
        \arrayrulecolor{blue}\hline
        A & B & C & D & E & F \\ [0.5ex] 
        \arrayrulecolor{blue}\hline
        \arrayrulecolor{blue}\hline
        1 & 2 & 3 & 4 & 5 & 6\\ 
        \arrayrulecolor{blue}\hline
        3 & 5 & 2 & \fcolorbox{red}{white}{4} & 1 & 6  \\ [1ex] 
        \arrayrulecolor{blue}\hline
        \end{tabular}
        \end{table}
    
\end{enumerate}

\end{document}
